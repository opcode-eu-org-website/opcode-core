% zbiór funkcji ułatwiajacych tworzenie pokazowych plików LaTeX

\newcommand{\pkgLink}[1]{\texttt{\href{https://ctan.org/pkg/#1}{#1}}}
\newcommand{\rrpPkgLink}[1]{\texttt{\href{https://bitbucket.org/OpCode-eu-org/latex-libs/}{#1}}}

\newenvironment{CatchExample}{
	\VerbatimEnvironment
	\begin{VerbatimOut}{\jobname.0.tmp}%
}{%
	\end{VerbatimOut}
}
\newenvironment{CatchExample*}[1]{
	\VerbatimEnvironment
	\begin{VerbatimOut}{\jobname.#1.tmp}%
}{%
	\end{VerbatimOut}
}
\newenvironment{MintedCode}{
	\VerbatimEnvironment
	\begin{adjustwidth}{.05\textwidth}{.05\textwidth}
	\begin{minted}[breaklines=false,breakafter={},numbers=left,numbersep=4pt,fontsize=\footnotesize]{tex}%
}{%
	\end{minted}
	\end{adjustwidth}
}

\let\FancyVerbFormatLineOrg\FancyVerbFormatLine
\renewcommand{\FancyVerbFormatLine}[1]{%
	\colorbox[rgb]{.9,.9,.9}{\hbox to .975\linewidth {#1}}%
}
\newcommand{\putExampleVerbatim}[1][0]{
	\inputminted[breaklines=false,breakafter={},numbers=left,numbersep=4pt,fontsize=\footnotesize]{latex}{\jobname.#1.tmp}
}
\newcommand{\putExampleVerbatimAdjust}[1][0]{
	\begin{adjustwidth}{.05\textwidth}{.05\textwidth} \putExampleVerbatim[#1] \end{adjustwidth}
}
\newcommand{\putExampleTeX}[1][0]{
	\input{\jobname.#1.tmp}
}

\newenvironment{Example}{
	\VerbatimEnvironment
	\begin{CatchExample}%
}{%
	\end{CatchExample}
	\nopagebreak
	
	\hspace{0.015\textwidth}\parbox[c]{0.6\textwidth}{%
		\vspace{-\topsep}\vspace{-\partopsep}\vspace{-\parskip}%
		\putExampleVerbatim%
		\vspace{-\topsep}\vspace{-\partopsep}%
	}%
	\hspace{0.03\textwidth}\parbox[c]{0.3\textwidth}{\centering%
		\putExampleTeX%
	}%
}

\newenvironment{Example*}{
	\VerbatimEnvironment
	\begin{CatchExample}%
}{%
	\end{CatchExample}
	\nopagebreak
	
	\hspace{0.015\textwidth}\parbox[c]{0.6\textwidth}{%
		\vspace{-\topsep}\vspace{-\partopsep}\vspace{-\parskip}%
		\putExampleVerbatim%
		\vspace{-\topsep}\vspace{-\partopsep}%
	}%
	\hspace{0.03\textwidth}\parbox[c]{0.3\textwidth}{%
		\putExampleTeX%
	}%
}

\newenvironment{ExampleVertical}{
	\VerbatimEnvironment
	\begin{CatchExample}%
}{%
	\end{CatchExample}
	\putExampleVerbatimAdjust\vspace{-0.5cm}%
	\begin{center}%
		\putExampleTeX%
	\end{center}%
}

\newenvironment{ExampleVertical*}{
	\VerbatimEnvironment
	\begin{CatchExample}%
}{%
	\end{CatchExample}
	\putExampleVerbatimAdjust%
	\putExampleTeX%
}

% definiujemy środowiska Example i ExampleVertical
%\makeatletter % powoduje że możemy używać @ w nazwach makr
 %\@namedef{Example}{\FV@Environment{}{VerbatimOut}{\jobname.tmp}}
 %\@namedef{endExample}{\@nameuse{FVE@VerbatimOut}\PrintCodeAndExampleA{\jobname.tmp}}
 %\@namedef{ExampleVertical}{\FV@Environment{}{VerbatimOut}{\jobname.tmp}}
 %\@namedef{endExampleVertical}{\@nameuse{FVE@VerbatimOut}\PrintCodeAndExampleB{\jobname.tmp}}
%\makeatother % przywracamy standardowe znaczenie @
