% Copyright (c) 2019-2020 Robert Ryszard Paciorek <rrp@opcode.eu.org>
% 
% MIT License
% 
% Permission is hereby granted, free of charge, to any person obtaining a copy
% of this software and associated documentation files (the "Software"), to deal
% in the Software without restriction, including without limitation the rights
% to use, copy, modify, merge, publish, distribute, sublicense, and/or sell
% copies of the Software, and to permit persons to whom the Software is
% furnished to do so, subject to the following conditions:
% 
% The above copyright notice and this permission notice shall be included in all
% copies or substantial portions of the Software.
% 
% THE SOFTWARE IS PROVIDED "AS IS", WITHOUT WARRANTY OF ANY KIND, EXPRESS OR
% IMPLIED, INCLUDING BUT NOT LIMITED TO THE WARRANTIES OF MERCHANTABILITY,
% FITNESS FOR A PARTICULAR PURPOSE AND NONINFRINGEMENT. IN NO EVENT SHALL THE
% AUTHORS OR COPYRIGHT HOLDERS BE LIABLE FOR ANY CLAIM, DAMAGES OR OTHER
% LIABILITY, WHETHER IN AN ACTION OF CONTRACT, TORT OR OTHERWISE, ARISING FROM,
% OUT OF OR IN CONNECTION WITH THE SOFTWARE OR THE USE OR OTHER DEALINGS IN THE
% SOFTWARE.

\newcommand{\teacherVersion}{TRUE}
\directlua{
	--[[ set srcDir (dirname of source file) and srcName (filename of source file without extention and --teacher surfix) --]]
	path,_      = string.gsub(status.filename, ".tex$", ""); --[[ $ --]]
	path,_      = string.gsub(path, "--teacher$", ""); --[[ $ --]]
	srcDir,_    = string.gsub(path, "[^/]*$", ""); --[[ $ --]]
	srcName,_   = string.gsub(path, "[^/]*/", "");
	_ = nil; path = nil;
	
	--[[ print info about srcDir and srcName --]]
	texio.write('Build „teacher” version for file „' .. srcName .. '” in „' .. srcDir .. '” directory');
	texio.write_nl('');
}

\newcommand{\rozwiazania}{
	\begin{teacherOnly}
		\typeout{Używam pliku z rozwiązaniami zadań: \directlua{ tex.sprint(srcDir .. 'rozwiazania/' .. srcName .. '.tex'); }}
		\InputIfFileExists{
			\directlua{ tex.sprint(srcDir .. 'rozwiazania/' .. srcName .. '.tex'); }
		}{}{}
	\end{teacherOnly}
}

\input{ \directlua{ tex.sprint(srcDir .. srcName .. '.tex'); } }
